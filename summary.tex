\chapter*{Summary}
\addcontentsline{toc}{chapter}{Summary}

In first chapter of this paper we have overviewed several
modern approaches to digital image-based steganography (LSB embedding, F5 embedding, complementary embedding)
and steganalysis (chi-square attack, S-family attack, attacks using higher order statistics
and attacks based on Benford's law). We have  ended this chapter with analysis of currently available
free applications and concluded, that most of them have many flaws.

Based on this knowledge, we have deduced a set of requirements to a steganographic application, 
that would eliminate most of these mistakes, in chapter 2.

In third chapter, we've described algorithms, used in the application. Then we've discussed
the possibility of using lossy channels to send images. We ended this chapter with a set of
``good policies'', that a user should follow to achieve maximum effectiveness of our application.

In fourth (and last) chapter of this paper, we've overviewed technical aspects of our application and 
described how to substitute used steganographic and cryptographic algorithms with new ones.

During our work, we've implemented complementary embedding steganographic algorithm in Java,
so future Android applications won't have to use inferior F5 library no more. We've also found a mistake in CE algorithm
description in the original paper, which we've fixed in our implementation (see chapter 3 for details).
We also developed a skeleton of an image steganography application, so future developers won't have to
implement JPEG compression and decompression algorithms by themselves.

\subsubsection{Future work}
Our application could be enhanced in several ways:
\begin{itemize}
  \item possibility of using asymmetric cryptography
  \item key management (something like a contact book)
  \item more eye-candy interface
  \item possibility to send cover images through lossy channels (briefly discussed in third chapter)
  \item self-testing with modern steganalytic tools
  \item running the calculations as background processes
  \item in-app possibility to send cover images (e.g. ``Share'' button)
\end{itemize}
