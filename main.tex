\documentclass[12pt, oneside]{book}
\usepackage[a4paper,top=2.5cm,bottom=2.5cm,left=3.5cm,right=2cm]{geometry}
\usepackage[utf8]{inputenc}
\usepackage[T1]{fontenc}
\usepackage{graphicx}
\usepackage{url}
%\usepackage[slovak]{babel} % vypnite pre prace v anglictine
\linespread{1.25} % hodnota 1.25 by mala zodpovedat 1.5 riadkovaniu

% my includes
\usepackage{color}
\usepackage{tocloft}
\usepackage{comment}
\usepackage{moreverb}

% -------------------
% --- Definicia zakladnych pojmov
% --- Vyplnte podla vasho zadania
% -------------------
\def\mfrok{2016}
%\def\mfnazov{Názov vašej bakalárskej práce}
%\def\mftyp{Bakalárska práca}
%\def\mfautor{Meno Priezvisko, príp. tituly}
%\def\mfskolitel{tit. Meno Priezvisko, tit. }
\def\mfnazov{Image-based steganography using a~mobile~phone}
\def\mftyp{Bachelor's thesis}
\def\mfautor{Askar Gafurov}
\def\mfskolitel{RNDr. Michal Forišek, PhD.}

%ak mate konzultanta, odkomentujte aj jeho meno na titulnom liste
\def\mfkonzultant{tit. Meno Priezvisko, tit. }  

\def\mfmiesto{Bratislava, \mfrok}

%aj cislo odboru je povinne a je podla studijneho odboru autora prace
%\def\mfodbor{2508 Informatika} 
%\def\program{ Informatika }
%\def\mfpracovisko{ Katedra informatiky }
\def\mfodbor{2508 Computer Science} 
\def\program{ Computer Science }
\def\mfpracovisko{ Department of Computer Science }

%% my macros
\def\TODO{\textbf{\textcolor{red}{TODO }}}

\begin{document}     

% -------------------
% --- Obalka ------
% -------------------
\thispagestyle{empty}

\begin{center}
\sc\large
%Univerzita Komenského v Bratislave\\
%Fakulta matematiky, fyziky a informatiky
Comenius University in Bratislava\\
Faculty of Mathematics, Physics and Informatics


\vfill

{\LARGE\mfnazov}\\
\mftyp
\end{center}

\vfill

{\sc\large 
\noindent \mfrok\\
\mfautor
}

\eject % EOP i
% --- koniec obalky ----

% -------------------
% --- Titulný list
% -------------------

\thispagestyle{empty}
\noindent

\begin{center}
\sc  
\large
%Univerzita Komenského v Bratislave\\
%Fakulta matematiky, fyziky a informatiky
Comenius University in Bratislava\\
Faculty of Mathematics, Physics and Informatics

\vfill

{\LARGE\mfnazov}\\
\mftyp
\end{center}

\vfill

\noindent
\begin{tabular}{ll}
%Študijný program: & \program \\
%Študijný odbor: & \mfodbor \\
%Školiace pracovisko: & \mfpracovisko \\
%Školiteľ: & \mfskolitel \\
Study program: & \program \\
Branch of studies: & \mfodbor \\
%Školiace pracovisko: & \mfpracovisko \\
Supervisor: & \mfskolitel \\
% Konzultant: & \mfkonzultant \\
\end{tabular}

\vfill


\noindent \mfmiesto\\
\mfautor

\eject % EOP i


% --- Koniec titulnej strany


% -------------------
% --- Zadanie z AIS
% -------------------
% v tlačenej verzii s podpismi zainteresovaných osôb.
% v elektronickej verzii sa zverejňuje zadanie bez podpisov

\newpage 
\thispagestyle{empty}
\hspace{-2cm}\includegraphics[width=1.1\textwidth]{images/zadanie}

% --- Koniec zadania

\frontmatter

% -------------------
%   Poďakovanie - nepovinné
% -------------------
\setcounter{page}{3}
\newpage 
~

\vfill
%{\bf Poďakovanie:}
{\bf Acknowledgements:}

I'd like to thank my supervisor RNDr. Michal Forišek, PhD. 
for ignoring my e-mails and thus cotributing to my independence
and character development.

\vfill

% --- Koniec poďakovania

% -------------------
%   Abstrakt - Slovensky
% -------------------
\newpage 
\section*{Abstrakt}


Slovenský abstrakt v rozsahu 100-500 slov, jeden odstavec. Abstrakt
stručne sumarizuje výsledky práce. Mal by byť pochopiteľný pre bežného
informatika. Nemal by teda využívať skratky, termíny alebo označenie
zavedené v práci, okrem tých, ktoré sú všeobecne známe.

\TODO Abstrakt v slovencine

\paragraph*{Kľúčové slová:} steganografia, Android, open-source, JPEG, komplementárne vkládanie 
% --- Koniec Abstrakt - Slovensky


% -------------------
% --- Abstrakt - Anglicky 
% -------------------
\newpage 
\section*{Abstract}

\TODO abstract in English

\paragraph*{Keywords:} steganography, Android, open-source, JPEG, complementary embedding

% --- Koniec Abstrakt - Anglicky

% -------------------
% --- Predhovor - v informatike sa zvacsa nepouziva
% -------------------
%\newpage 
%\thispagestyle{empty}
%
%\huge{Predhovor}
%\normalsize
%\newline
%Predhovor je všeobecná informácia o práci, obsahuje hlavnú charakteristiku práce 
%a okolnosti jej vzniku. Autor zdôvodní výber témy, stručne informuje o cieľoch 
%a význame práce, spomenie domáci a zahraničný kontext, komu je práca určená, 
%použité metódy, stav poznania; autor stručne charakterizuje svoj prístup a svoje 
%hľadisko. 
%
% --- Koniec Predhovor


% -------------------
% --- Obsah
% -------------------

\newpage 

\tableofcontents

% ---  Koniec Obsahu

% -------------------
% --- Zoznamy tabuliek, obrázkov - nepovinne
% -------------------

\newpage 

\listoffigures

% ---  Koniec Zoznamov

\mainmatter

\chapter*{Introduction}
\addcontentsline{toc}{chapter}{Introduction}

\TODO

%\part{EXISTED KNOWLEDGE}

\input steganography.tex

%\part{OUR WORK}

\input goals.tex

\chapter{Theoretical aspects of application}

In this chapter we will talk about steganographic communication protocol,
discuss various possible communication channels (carriers), describe
used algorithms (both for steganography and cryptography) with concrete
settings. Then we will talk about limits of algorithms we've chosen and
how to maximize their strength by follow "good policies" of usage.

\section{Communication protocol}

\TODO diagram

\section{Analysis of communication channels}

\paragraph{Twitter}
\paragraph{Facebook}
\paragraph{Instagram}
\paragraph{Google+}
\paragraph{Google Photos}
\paragraph{4chan.org}
\paragraph{9gag.com}
\paragraph{reddit.com}



\section{Used algorithms}

\subsection{Compression algorithm}
\TODO ZIP - concrete constants and link to detailed description.
\subsection{Encryption algorithm}
\TODO AES - concrete constants and link to detailed description.
\subsection{Steganographic algorithm}
\TODO probably E1 - concrete constants and reference to first chapter with detailed explanation.

\section{Usage of application}
\subsection{Secret keys}
\subsection{Limits of message length}
\subsection{Optimal conditions for undetectability}

\chapter{Implementation}

In this chapter we will talk about interesting implmentation details of
our application.

\section{Design of interfaces for interchangable libraries}

\section{Android phone capabilities}


 
%\part{OLD STUFF}
%\input old_stuff.tex

\chapter*{Summary}
\addcontentsline{toc}{chapter}{Summary}
\TODO

% -------------------
% --- Bibliografia
% -------------------


\newpage	

\backmatter

\thispagestyle{empty}
\nocite{*}
\clearpage

\bibliographystyle{plain}
\bibliography{literatura} 


% -------------------
%--- Prilohy---
% -------------------

%Nepovinná časť prílohy obsahuje materiály, ktoré neboli zaradené priamo  do textu. Každá príloha sa začína na novej strane.
%Zoznam príloh je súčasťou obsahu.
%
%\addcontentsline{toc}{chapter}{Appendix A}
%\input manual.tex
%
%\addcontentsline{toc}{chapter}{Appendix B}
%\input AppendixB.tex

\end{document}
