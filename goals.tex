\chapter{Goal of this work}
\label{ch:goal}

In this chapter we will specify requirements on our application.

\section{Motivation}

Our initial motivation was to create an application for communication between
people in countries with censored Internet (e.g. Syria, Egypt), with only limited communication channels
available (e.g. e-mails, Facebook, ICQ, etc.). Particularly we are interested in
creating a way to communicate with outer world securely and inconspicuously 
(e.g. correspondents).

We have chosen digital images as our communication medium because it is only natural for
a mobile phone to have plenty of almost useless photos (e.g. photos of someone's dog or dinner)
and these photos are being sent through various social media all the time.

In chapter \ref{sec:image-steganography} we discussed about different formats for
this purpose, and we've chosen \emph{JPEG} as the most suitable.

\section{Functional requirements}
In this section we will discuss \emph{which} tasks our application has to be able to perform.

\subsubsection{Embedding and extracting of messages into images}
Main function is to embed text message into JPEG files from the phone's camera using steganography.

The steganographic algorithm must be key-based, as for in the opposite case it would be very for an attacker
to extract the hidden message (you can read more on this in \cite{Ostertag1996}).

The application should forbid to use images from phone's memory, for it could lead to multiple 
using of the same image, which is \emph{very bad} for user's inconspicuousness.

\subsubsection{Message encryption}
The application has to be able to encrypt and decrypt the message. This also contributes
to undetectability of the secret message, as the result of extracting still would be 
indistinguishable from a random noise.

We also have had to decide whether we will use symmetric or asymmetric
encryption. We decided to stick with symmetric, as it has easier key management.
Also, if a user would like to send a message to many people, he would have to
send the same message many times with different keys, whether embedded in one 
image (which would make the detection easier) or in many images (which would
also increase suspicion).

\subsubsection{Message compression}
The secret message has to compressed in order to take minimum possible space in medium,
as bigger changes in medium are easier to detect.

\subsubsection{Key management}
Our application has to be able to generate secure keys for steganographic and
cryptographic algorithms. It is also should be possible to share a key with someone
else (in order to communicate).

Whenever an application has to deal with keys, it is a good idea to conceal them
and reveal them only by entering a so called \emph{master key}. We decided not to demand
this feature, as if a user and his mobile phone are already in hands of the censors, it is only 
a matter of time when he would give up his master key. The whole purpose of our application
is to avoid the contact with the censors.

\subsubsection{Optimal conditions of undetectability}
Our application has to be able to decide whether it is sufficiently secure to 
send a message of given length. 



\section{Technical requirements}
In this section we will discuss requirements on \emph{how} our application has
to perform its tasks.

\subsubsection{Support of UTF-8}
Our application is meant to be used all around the world, and not all languages use
Latin alphabet to encode it (e.g. Arabic, Russian). 

\subsubsection{Interchangeability of algorithms}
As steganography and steganalysis develop very quickly, our application
has to have the ability to easily change used algorithms.  

\subsubsection{Openness of the code}
No one can trust an application that has closed source code (not only because
this code could be harmful, but also because no one can check whether all
methods were implemented correctly).

\subsubsection{Ability to withstand additional compression}
Many social media are known to compress any images that are sent through
their systems. We'd like to analyze whether it is possible to save the information
in the image instead of compression (e.g. by using error correction encoding).
