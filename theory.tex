\chapter{Theoretical aspects of application}
In this chapter we will takl about the communication protocol,
describe used algorithms (with an accent on steganography), 
overview various possible communication channels (carriers).
% TODO remove if won't be.
We will also talk briefly about resistance of used algorithms
against additional compression of image with embedded message.
We will end this chapter with a list of ''good practices''
a user should follow to achieve maximum possible security
and undetectability.


\section{Communication scheme}
In this section we will describe how the communication works.

% TODO insert the communication diagram

\section{Used algorithms}
In this section we will describe algorithms that we used in our application
and discuss their suitness for our purpose. 
We have had to find a balance between potential quality of the algorithms 
and the cost of their implementation and usage.

% TODO find a better way to say it?
\textit{We'd like to mention that these algorithms are not the product of our work.}

\subsection{Steganographic algorithm: Complementary embedding}
% TODO link to the whole paper

This algorithm was proposed by Chiang-Lung~Liu and Shiang-Rong~Liao in \cite{liu2008high}.
We've chosen this algorithm as it has better steganographic properties than the ''unofficial
stadnart'' algorithm F5 and is relatively simple to implement and to verify its correctness.

This algorithm was primarily developed to withstand S-attacks (as opposed to F5~algorithm).
It achieves this goal by dividing both message and DCT~coefficients into two parts
(according to steganographic~key and predefined separation~ratio~$\alpha$) and
modifying coefficients in the first part by subtracting and in the second part by addition.

\subsubsection{Embedding process}
Input for the algorithm:

\begin{itemize}
    \item $K$ steganographic key (seed for random generator)
    \item $B$ bytes to be embedded (secret message, encrypted and compressed)
    \item $\alpha$ separation ratio
    \item $\beta$ adjustment parameter
    \item $D$ quantized DCT coefficients sequence
\end{itemize}

Output of the algorithm:

\begin{itemize}
    \item $D'$ changed quantized DCT coefficients
\end{itemize}

\begin{Algo}
\item 
Use a stego-key $K$ to create a permutation of quantized coefficients $D$.
That is, $$Q := PERM_K(D),$$ where $PERM(\cdot)$ is a key-dependent
permutation function and $Q$ denotes the permuted coefficient sequence.
\item
Divide $Q$ into two parts, $Q_1$ and $Q_2$, according to separation ratio $\alpha$.
That is, $$Q_1 := Q[0 : \alpha * LEN(Q)],~ Q_2 := Q[\alpha * LEN(Q) : LEN(Q)],$$
where $LEN(\cdot)$ denotes the length and $Q[a : b]$ denotes a slice of an array from 
$a$-th to $b$-th element (half-open interval, i.e. $b$-th element does not belong into slice).
\item 
Separate the secret message bytes sequence $B$ into to two parts $B_1$ and $B_2$, 
according to the separation ratio $\alpha$ (similarly to previous step).
\item
Let $L_1$ and $L_2$ denote the byte representations of $LEN(B_1)$ and $LEN(B_2)$, respectively.
Let $M_1$ be the concatenation of the $L_1$ and $B_1$ ($M_1 := L_1 || M_1$). Similarly, let
$M_2 := L_2 || B_2$. That would be our two parts we will embed into the cover image.
\item
Convert byte sequences $M_1$ and $M_2$ to bit sequences $M'_1$ and $M'_2$ (by using little endian
encoding).
\item
Embed $M'_1$ into the non-zero coefficients of $Q_1$ using algorithm E1. Note that
the adjustment parameter $\beta$ is used here to make it more resistable against statistical attacks.
\item
Embed $M'_2$ into the non-zero coefficients of $Q_2$ using algorithm E2.
\item
Combine $Q_1$ and $Q_2$ to form a single coefficient sequence $Q'$.
\item 
Using stego-key $K$, depermute the coefficient sequence $Q'$ into $D'$. That is,
$$D' := PERM^{-1}_K(Q'),$$ where $PERM^{-1}$ denotes the inverse permutation.
\item 
Return $D'$.
\end{Algo}

Algorithms E1 and E2 are essentially just for-loops through all bits of the secret message
and coefficients with a complex if-statement inside, descibing how to change coefficients
(changes of coefficients depend on oddness of the coefficient and
value of the secret bit). Detailed pseudocode can be found in the original paper.

Here we show our implementation of these algorithms' if-statements 
in a single function (in Java programming language). 
Variable~$type$ denotes whether we are using E1 or E2 algotithm.

\begin{lstlisting}
public int embedBitUnsafe (final int type, final int c, final int bit) {
    // type: 1 for E1, 2 for E2 algorithm
    // c: (non-zero) coefficient we are changing
    // bit: secret bit to embed
    if (type == 1) {
        if (c > 0 && isOdd(c)) {
            if (bit == 0 && c-1 == 0) return c-2;
            else if (bit == 0 && c-1 != 0) return c-1;
            else if (bit == 1) return c;
        }
        else if (c > 0 && isEven(c)) {
            if (bit == 1) return c-1;
            else if (bit == 0) return c;
        }
        else if (c < 0 && isOdd(c)) {
            if (bit == 1) return c-1;
            else if (bit == 0) return c;
        }
        else if (c < 0 && isEven(c)) {
            if (bit == 0) return c-1;
            // in paper the condition is (c == 1), but it doesn't make much sense
            else if (bit == 1) return c;
        }
    }
    else if (type == 2) {
        if (c > 0 && isOdd(c)) {
            if (bit == 1) return c+1;
            else if (bit == 0) return c;
        }
        else if (c > 0 && isEven(c)) {
            if (bit == 0) return c+1;
            else if (bit == 1) return c;
        }
        else if (c < 0 && isOdd(c)) {
            if (bit == 0 && c + 1 == 0) return c+2;
            else if (bit == 0 && c + 1 != 0) return c+1;
            else if (bit == 1) return c;
        }
        else if (c < 0 && isEven(c)) {
            if (bit == 1) return c+1;
            else if (bit == 0) return c;
        }
    }
}
\end{lstlisting}

Algorithm E1 additionally change a part of used coefficients using the adjustment
parameter $\beta$. The adjustment is simple: we need to look at first $\beta * LEN(M)$ coefficients and
change values $-2$ to value $1$.

\subsubsection{Extraction process}
Input for the algorithm:

\begin{itemize}
    \item $K$ steganographic key (seed for random generator)
    \item $\alpha$ separation ratio
    \item $\beta$ adjustment parameter
    \item $D$ quantized DCT coefficients sequence with embedded message
\end{itemize}

Output of the algorithm:

\begin{itemize}
    \item $B$ extracted bytes of the secret message
\end{itemize}

\begin{Algo}
\item 
Use a stego-key $K$ to create a permutation of quantized coefficients $D$.
That is, $$Q := PERM_K(D),$$ where $PERM(\cdot)$ is a key-dependent
permutation function and $Q$ denotes the permuted coefficient sequence.
\item
Divide $Q$ into two parts, $Q_1$ and $Q_2$, according to separation ratio $\alpha$.
That is, $$Q_1 := Q[0 : \alpha * LEN(Q)],~ Q_2 := Q[\alpha * LEN(Q) : LEN(Q)],$$
where $LEN(\cdot)$ denotes the length and $Q[a : b]$ denotes a slice of an array from 
$a$-th to $b$-th element (half-open interval, i.e. $b$-th element does not belong into slice).
\item
Extract the length $L_1$ of the embedded message from the $Q_1$, then extract $L_1 * 8$ bits of
the embedded message $B'_1$ from $Q_1$ by using the following algorithm:
\begin{lstlisting}
if (c > 0 && isEven(c)) return 0;
if (c < 0 && isOdd(c))  return 0;
if (c > 0 && isOdd(c))  return 1;
if (c < 0 && isEven(c)) return 1;
\end{lstlisting}
\item
Similarly, extract the length $L_2$ of the embedded message from $Q_2$ and $L_2 * 8$ bits of the 
embedded message $B'_2$ by using the following algorithm:
\begin{lstlisting}
if (c > 0 && isOdd(c))  return 0;
if (c < 0 && isEven(c)) return 0;
if (c > 0 && isEven(c)) return 1;
if (c < 0 && isOdd(c))  return 1;
\end{lstlisting}
\item
Convert bit sequences of the secret messages $B'_1$ and $B'_2$ to byte sequences $B_1$ and $B_2$.
\item
Combine $B_1$ and $B_2$ into single secret message $B$.
\item 
Return $B$.

\end{Algo}

\subsubsection{Tuning the parameters $\alpha$ and $\beta$}
The original paper suggests that the separation ratio $\alpha$
should be chosen from a~segment~$\left[ \frac{3}{4}, \frac{5}{6} \right]$
and the adjustment parameter $\beta$ should be chosen from a~segment~ $\left[\frac{1}{8}, \frac{1}{3}\right]$ 
(bigger $\beta$ positively correlates with
better undetectability of the message). 
In tests the authors have chosen
values $\frac{4}{5}$ and $\frac{1}{3}$, respectively.

\subsubsection{Limit for a message length}
The authors of the algorithm implies that for chosen values of the parameters
it is safe enough to use even $50\%$ of nonzero coefficients, i.e. the algorithm
is still resistable to chi-square and S families of attacks. As for general measures
of changing of the cover image, authors tell us that it is safe enough to use all
nonzero coefficients available.

\subsection{Cryptographic algorithm: AES}
% TODO description and link to more specific paper
% TODO add link to implementation
% TODO add link to the original paper

\subsection{Compression algorithm: GZIP}
% TODO link to original paper or standard


\section{Resistance against additional compression}
% TODO it is bad :)


\section{Communication channels}
In this section we will enlist communication channels that are suitable for our purpose.
% TODO really? :D

\section{How to use the application correctly}
In this section we will describe common mistakes and give some advices on
how to achieve maximum inconspicuousness.

\paragraph{Never use the same image twice}

\paragraph{Never send the same message twice}

\paragraph{Change your keys frequently}

\paragraph{Use small images}
% TODO or rather bigg? vivid? black-white?

\paragraph{Try to write your messages as short as possible}

\paragraph{Generate your keys by app's utilities, not by hand}

% TODO something about key management?
