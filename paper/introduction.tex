\chapter*{Introduction}
\addcontentsline{toc}{chapter}{Introduction}


Steganography is ``an art of secret writing''. Its goal is 
to conceal the fact of communication. It is often compared
to cryptography, which purpose is to conceal the information,
carried by the message. Steganography is probably as old as written
communication itself. In ancient times, people wrote their messages
under layer of wax of wax tablets to sneak their messages into another
city, or used invisible inks (e.g. milk) to write their notes.

In modern days, digital steganography has arisen. Messages are embedded
into digital texts or multimedia (like images or music).

Digital steganography and steganalysis (art of revealing hidden messages) 
are young branches of modern science and new approaches are developed each year.

Despite its growing popularity, many applications are still using old methods,
which have been proven to be unreliable. Applications for mobile phones, specifically,
are very vulnerable because of usage of old algorithms and limited knowledge of their
developers about modern state of the area.

In this paper, we will discuss modern methods of steganography and steganalysis, analyze
the common mistakes of currently available applications (chapter \ref{ch:stego}). Then we
will develop our own free open-source application for steganography. Firstly, we will specify
functional and technical requirements on our application. Application would be created in a way
that would encourage users to avoid mistakes and achieve maximum security and inconspicuousness
(chapter \ref{ch:goal}). Secondly, we will discuss theoretical aspects of our application: describe
used algorithms and enlist ``good practices'' that could help a user (chapter \ref{chap:theory}).
Lastly, we will describe technical details of our application (chapter \ref{ch:imple}).
